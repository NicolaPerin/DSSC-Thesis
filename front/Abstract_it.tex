\chapter{Abstract, italian}

Gli esperimenti di microscopia elettronica (EM) generano dataset molto grandi e complessi. Questi sono spesso memorizzati in formati specifici del fornitore con pochi metadati, il che li rende difficili da condividere, riprodurre o riutilizzare. Per affrontare questo problema, questa tesi applica i principi FAIR---Findable, Accessible, Interoperable e Reusable---ai flussi di lavoro del Laboratorio di Microscopia Elettronica (LAME) di Trieste, Italia.

Il progetto sviluppa e testa un'infrastruttura dati \emph{FAIR-by-design} che:
\begin{itemize}
	\item cattura dati e metadati direttamente durante l’acquisizione,
	\item li converte e archivia automaticamente in formato standardizzato NeXus/NXem,
	\item utilizza un data lake centralizzato per garantire durabilità e scalabilità,
	\item e fornisce un’applicazione web per caricamenti strutturati, annotazioni e accesso.
\end{itemize}

Il sistema separa le operazioni di microscopia dall’elaborazione intensiva, affidandosi al centro di calcolo ad alte prestazioni ORFEO per archiviazione e computazione. Un ambiente di test virtuale è stato utilizzato per sviluppare e validare in sicurezza lo stack software prima del rilascio in produzione. Questo garantisce che i dati EM vengano preservati in formati interoperabili, arricchiti con metadati e pronti per l’accesso aperto.

Il lavoro contribuisce con un modello pratico per una gestione dei dati FAIR e riproducibile in microscopia, con il potenziale di scalare all’interno del programma NFFA-DI e allinearsi alle politiche europee di \emph{open science}.

