\chapter{Abstract, italian}

Gli esperimenti di microscopia elettronica (EM) generano dataset molto grandi e complessi. 
Questi sono spesso memorizzati in formati proprietari con poche informazioni descrittive (metadata), 
il che rende difficile condividerli, riprodurli o riutilizzarli.  
Per affrontare questo problema, questa tesi applica i principi FAIR---Findable, Accessible, Interoperable, Reusable---ai flussi di lavoro del \textit{Laboratorio di Microscopia Elettronica} (LAME), parte dell’Istituto di \textit{Ricerca e Innovazione Tecnologica} (RIT) presso Area Science Park a Trieste.

Il progetto sviluppa e testa un’infrastruttura dati FAIR-by-design che:
\begin{itemize}
	\item cattura dati e metadata direttamente durante l’acquisizione,
	\item li converte e archivia automaticamente in formato standardizzato NeXus/NXem,
	\item utilizza un \textit{data lake} centralizzato per garantire durabilità e scalabilità,
	\item e fornisce un’applicazione web per caricamenti strutturati, annotazioni e accesso.
\end{itemize}

Il sistema separa le operazioni al microscopio dall’elaborazione intensiva, 
affidandosi al centro di calcolo ad alte prestazioni ORFEO per archiviazione e computazione.  
Un ambiente virtuale di test è stato usato per sviluppare e validare in sicurezza lo stack software prima del rilascio in produzione.  
Questo assicura che i dati EM vengano conservati in formati interoperabili, arricchiti con metadata e pronti per l’accesso aperto.

Il lavoro contribuisce con un modello pratico e innovativo per la gestione FAIR e riproducibile dei dati in microscopia, 
con il potenziale di scalare ad altre infrastrutture di ricerca in scienza dei materiali, come quelle affiliate al programma NFFA-DI\footnote{\url{https://nffa-di.it/it/about-us/project/}}, 
e di allinearsi alle politiche europee per la scienza aperta.
