\chapter{Deployment and Validation in the VirtualOrfeo Environment}\label{chap:virtualorfeo-deployment}

\section{VirtualOrfeo as a Digital Twin of the HPC Environment}

To develop and test the data management platform in a realistic but safe setting, we use an internal system called \textbf{VirtualOrfeo}. VirtualOrfeo functions as a \emph{digital twin} of the ORFEO HPC cluster: a virtualized replica of its key infrastructure and services. This allows us to deploy and evaluate applications under conditions that closely mimic production, without the risk of disrupting the actual cluster. Digital twins are often used in High-Performance Computing (HPC) to experiment with configuration changes or new workflows before rolling them out in production \parencite{Ohmura2023TwdsDT}.

VirtualOrfeo was originally created by Isac Pasianotto and Ruggero Lot as a general-purpose environment for ORFEO infrastructure testing. The work in this thesis builds on their platform: we use it to deploy and integrate the data management application, while the underlying environment itself is their contribution.

The system is built from a collection of virtual machines (VMs) that replicate the main roles of the HPC cluster. They run on Linux KVM/QEMU and are managed through Vagrant and Ansible. The virtual machines represent essential services rather than the full production scale. For example:  

\begin{itemize}
	\item An authentication VM provides directory and identity services.  
	\item One or more VMs host a lightweight Kubernetes cluster for containerized applications.  
	\item Optional VMs simulate HPC login nodes, compute nodes with a Slurm scheduler, and storage nodes.  
\end{itemize}

The goal is to preserve fidelity where it matters while keeping the environment manageable. For instance, VirtualOrfeo runs a small Ceph cluster to mirror the storage architecture of ORFEO, and it includes a simplified Kubernetes deployment for cloud-native services. Differences remain: the virtual system operates at smaller scale, lacks specialized hardware, and cannot match the performance capacity of the real cluster. Still, the software stack and APIs are close enough to ensure that if our application runs successfully on VirtualOrfeo, it will work on the production cluster with only minor adjustments.

\medskip

In practice, VirtualOrfeo provides a controlled environment to connect our Django-based application with HPC components such as authentication, storage, and container orchestration. By offering a faithful but bounded replica of ORFEO’s software environment, it supports safe experimentation and rapid iteration, in line with DevOps practices in scientific computing \parencite{Ohmura2023TwdsDT}.

%%%%%%%%%%%%%%%%%%%%%%%%%%%%%%%%%%%%%%%%%%%%%%%%%%%%%%%%%%%%%%%%%%%%%%%%%%%%%%%%%%%%%%%%%%%%%%%

\section{K3s Cluster Topology in the Virtual Environment}

At the heart of VirtualOrfeo is a Kubernetes cluster deployed on one of the 
virtual machines (or across a couple of VMs) using \textbf{K3s}. K3s is a 
lightweight, certified Kubernetes distribution designed for resource-constrained 
or edge environments\parencite{Rancher2021K3s}. Unlike a full-fledged multi-node 
Kubernetes deployment, K3s packages the essential components into a single, 
easy-to-install binary, reducing the complexity and footprint of the cluster. In 
VirtualOrfeo, this K3s cluster provides the platform on which our web application 
and related services are orchestrated as containers.

\medskip

The cluster topology is relatively simple. In our setup:  
\begin{itemize}
	\item A single VM (\texttt{kube01}) acts as the Kubernetes control-plane node 
	and also runs as a worker node. This means the API server, scheduler, and 
	controller-manager run on that VM, and it also hosts the application pods. 
	(The K3s design allows one to add additional worker nodes easily if needed, 
	but one node is sufficient for testing purposes.)
	\item Another VM (\texttt{ipa01}) runs the identity management services 
	(discussed later).
	\item Additional VMs can be spun up for storage or other auxiliary roles 
	(e.g., Ceph OSD servers).
\end{itemize}

All these VMs are connected via a private virtual network so that services can 
reach each other using hostnames like \texttt{ipa01.virtualorfeo.it} or 
\texttt{kube01.virtualorfeo.it}.

\medskip

Within the K3s cluster, we adopt a logical separation of concerns using 
Kubernetes \textbf{namespaces}. For example, the Authentik identity provider is 
deployed in its own namespace (called \texttt{authentik}), and our Django 
application is deployed in another namespace (e.g.\ \texttt{lame-fair} for our 
``FAIR data'' app). Namespaces provide isolation for configuration and secrets, 
and they help mirror the structure of a real cluster where different services 
might be managed by different teams.

\medskip

To allow external access to services running in K3s (which by default would only 
be reachable inside the cluster network), VirtualOrfeo makes use of a Kubernetes 
\textbf{Ingress} controller together with MetalLB for load-balancing. The ingress 
controller (NGINX in our case) listens on a cluster IP address and routes 
incoming HTTP/HTTPS requests to the appropriate service based on hostname. 

MetalLB is a software load-balancer that can allocate a virtual IP address to 
the ingress controller on the local network. In the virtual environment we 
reserve an IP (for example, 192.168.132.100) as the ingress IP for all services. 
We then define DNS entries (or simply \texttt{/etc/hosts} mappings during 
development) such as:  
\begin{itemize}
	\item \texttt{auth.k3s.virtualorfeo.it}  
	\item \texttt{lame-fair.k3s.virtualorfeo.it}  
	\item \texttt{minio.k3s.virtualorfeo.it}  
\end{itemize}

All these hostnames point to the same ingress IP. This way, when we access 
\texttt{https://lame-fair.k3s.virtualorfeo.it} in a browser, the request hits the 
NGINX ingress on the K3s VM, which forwards it to the correct Kubernetes Service 
(our Django app service) within the cluster. 

The use of host-based routing and TLS termination at the ingress level mimics 
how services would be exposed in a production environment (with real DNS 
entries). The certificates used for TLS in VirtualOrfeo are issued by a local 
Certificate Authority (the FreeIPA server, acting as a CA), so that we have 
proper HTTPS without needing internet access. We configured 
\textbf{cert-manager} on the cluster to obtain and refresh certificates from 
this internal CA via the ACME protocol, automating the issuance of TLS 
certificates for all service hostnames.

\medskip

In terms of storage for the cluster, K3s provides a default storage class (built 
on local disk storage via the ``local-path'' provisioner), which we use for most 
purposes. For example, the PostgreSQL database backing our Django app uses a 
PersistentVolumeClaim that by default binds to local storage on 
\texttt{kube01}. This is acceptable in a virtual environment where data 
persistence is not mission-critical. 

Optionally, VirtualOrfeo’s Ceph deployment could be integrated via a Container 
Storage Interface (CSI) driver to supply distributed block storage (RBD volumes) 
or shared filesystem volumes (CephFS) to the cluster, thereby simulating a more 
production-like storage backend. However, to keep things simple, our deployment 
mostly relies on the out-of-the-box storage provisioner for now. The key storage 
integration we do leverage is object storage (Ceph’s RADOS Gateway), which will 
be discussed in Section~\ref{sec:virtualorfeo-storage}.

\medskip

To summarize the topology: VirtualOrfeo’s Kubernetes is a lightweight cluster 
hosting multiple namespaces (for Authentik, the Django app, monitoring tools, 
etc.), all accessible through an NGINX ingress on a single IP. The infrastructure 
pieces such as load balancing and certificate management are in place so that 
each service behaves as if it were in a normal production cluster with distinct 
hostnames and TLS. This setup allows our application to be deployed with the 
same Helm charts and configurations that we would use on a real cluster, 
providing high confidence that the behavior will be consistent when transitioning 
to production.


\medskip
\noindent With the cluster topology in place, we can turn to the auxiliary services 
that make the environment resemble a production HPC setting. 
We begin with identity and access management, which provides 
centralized authentication across the system.

%%%%%%%%%%%%%%%%%%%%%%%%%%%%%%%%%%%%%%%%%%%%%%%%%%%%%%%%%%%%%%%%%%%%%%%%%%%%%%%%%%%%%%%%%%%%%

\section{Identity and Access Management in VirtualOrfeo} \label{sec:virtualorfeo-iam}

The VirtualOrfeo environment includes a full \textbf{Identity and Access Management (IAM)} stack to mirror the authentication system of the production cluster. In this setup, we combine \textbf{FreeIPA} for directory services (accounts and groups) with \textbf{Authentik} as an OpenID Connect (OIDC) identity provider. 

FreeIPA is an open-source identity management solution that provides an LDAP directory and a Kerberos realm for centralized authentication. It can be considered the upstream community version of Red Hat Identity Manager. Authentik is an open-source identity provider that supports single sign-on using modern protocols such as OAuth2, OIDC, and also SAML.  

Together, these two systems simulate an enterprise-style authentication service. FreeIPA serves as the authoritative source of user identities (through LDAP), while Authentik acts as the user-facing login portal and token issuer for web applications.

\medskip

In the VirtualOrfeo setup, the FreeIPA server runs on the VM \texttt{ipa01}. It maintains a directory of user accounts and groups. For testing, dummy users can be created in FreeIPA (the Ansible scripts even set up some by default). FreeIPA also doubles as a Certificate Authority, issuing TLS certificates for services.  

On the Kubernetes (K3s) side, Authentik is deployed via Helm in the \texttt{authentik} namespace. After deployment, a one-time initial configuration is required:  

\begin{itemize}
	\item \textbf{Bind account for LDAP access:}  
	We create a dedicated service account in FreeIPA (e.g.\ \texttt{svc\_authentik}) for Authentik to use when binding to LDAP. This account is granted read permissions on users and groups (we added it to the ``User Administrators'' role). Authentik uses this account to synchronize users and to check passwords during login.  
	
	\item \textbf{LDAP source configuration in Authentik:}  
	In the Authentik admin interface, we add an LDAP Directory source pointing to the FreeIPA server. We configure:  
	\begin{itemize}
		\item the LDAPS URL (\texttt{ldaps://ipa01.virtualorfeo.it}),  
		\item the Base DN (e.g.\ \texttt{dc=virtualorfeo,dc=it}),  
		\item the Bind DN (\texttt{uid=svc\_authentik,cn=users,cn=accounts,...}) and password,  
		\item the attribute mapping (username $\rightarrow$ \texttt{uid}, email $\rightarrow$ \texttt{mail}, full name, groups).  
	\end{itemize}  
	Authentik ships with templates for FreeIPA/389ds, which simplifies this step. We enabled both user and group synchronization, and then triggered an initial sync: all existing FreeIPA users and groups were imported into Authentik.  
	
	\item \textbf{Federation of FreeIPA with Authentik:}  
	Once this sync is configured, any FreeIPA user can log into Authentik. Credentials are checked against FreeIPA, while Authentik manages sessions and tokens. Effectively, FreeIPA remains the backend store, and Authentik acts as the OIDC-facing layer.  
	
	\item \textbf{Registering the Django app as an OIDC client:}  
	We register our Django application in Authentik by creating a new Application entry (``Lame Fair'') and associating it with an OAuth2/OIDC Provider. Configuration details include:  
	\begin{itemize}
		\item \textbf{Client type:} Confidential (the app can securely store a client secret).  
		\item \textbf{Redirect URIs:}  
		\begin{itemize}
			\item \texttt{https://lame-fair.k3s.virtualorfeo.it/oidc/callback/}
			\item \texttt{https://lame-fair.k3s.virtualorfeo.it/}
		\end{itemize}
		\item \textbf{Flow:} Authorization Code flow with explicit user consent (OIDC default).  
	\end{itemize}
	
	Authentik then generates a Client ID and Client Secret, which we inject into the Django deployment as Kubernetes secrets.  
	
	\item \textbf{Scopes and claims:}  
	We enabled the standard \texttt{openid} scope plus \texttt{email}, \texttt{profile}, and \texttt{offline\_access}. This ensures the ID token includes email and profile data, and allows refresh tokens for long-lived sessions.  
	We also configured claims mapping so that the ID token carries \texttt{preferred\_username} and \texttt{email}. Group memberships could also be mapped to a claim (e.g.\ \texttt{groups}) to drive role-based access in the app. While not strictly required here, the data is synchronized and could support advanced authorization (e.g.\ marking members of a ``app-admins'' group as administrators).  
\end{itemize}

\medskip

Once configured, the login flow for our web application works as follows:  
\begin{itemize}
	\item The Django app redirects the user to Authentik’s OIDC authorization endpoint.  
	\item Authentik displays a login form and validates credentials against FreeIPA.  
	\item If successful, Authentik issues an authorization code and returns it to the app’s callback URL.  
	\item The Django app exchanges the code for an ID Token (and optionally an Access Token).  
	\item The ID Token is a JSON Web Token (JWT), signed by Authentik, containing identity claims such as username and email \parencite{OpenIDConnectCore}.  
	\item Django verifies the token using Authentik’s public key and logs the user in (creating a local session or record if necessary).  
\end{itemize}

From this point on, the user is authenticated in the Django app. Credentials are never handled by the application itself; authentication is delegated to Authentik and FreeIPA. This improves security, enables single sign-on across services, and ensures that disabling a FreeIPA account immediately prevents further access to the application.

\medskip

The VirtualOrfeo IAM configuration demonstrates the architecture outlined earlier in Section~\ref{sec:virtualorfeo-iam-design}, now realized in practice. We have established a trust between the Django application and Authentik via OIDC, and another trust between Authentik and FreeIPA via LDAP. FreeIPA remains the central source of truth for credentials, while Authentik provides the web-facing authentication layer and token issuance.  

This approach can be directly replicated in production. Other identity providers (such as Keycloak) could substitute for Authentik, but the principle remains the same: keep FreeIPA as the authoritative store, and place a modern IdP in front to provide OIDC/OAuth-based authentication for applications.

\medskip
\noindent Once authentication and identity services are integrated, 
the next step is to ensure that applications can interact 
with persistent storage in a reliable and secure way. 
We therefore examine how VirtualOrfeo incorporates object storage 
to mirror the production environment.

%%%%%%%%%%%%%%%%%%%%%%%%%%%%%%%%%%%%%%%%%%%%%%%%%%%%%%%%%%%%%%%%%%%%%%%%%%%%%%%%%%%%%%

\section{Storage Integration in VirtualOrfeo} \label{sec:virtualorfeo-storage}

Storage of experimental data is a major concern for our application, and 
VirtualOrfeo replicates the production storage environment to a large extent. In 
the real ORFEO facility, high-volume data (like images from microscopes) might 
be stored on a distributed file system or an object storage service for 
durability and scalability. In VirtualOrfeo, we deploy the \textbf{Ceph} storage 
system to provide a multi-purpose storage backend. Ceph is a widely used 
open-source distributed storage platform that can provide object storage, block 
storage, and file system interfaces within a unified cluster 
\parencite{Weil2006Ceph}. We focus on Ceph’s object storage interface, which is 
provided by the \textbf{RADOS Gateway (RGW)} component. The RGW exposes a 
RESTful API compatible with Amazon S3, meaning our application can interact with 
it as if it were talking to AWS S3, but the data is stored on-premise in Ceph.

\medskip

In VirtualOrfeo, a small Ceph cluster is set up using additional VMs (for monitor 
and OSD daemons). Once Ceph is running, an RGW daemon is deployed (in our case, 
on one of the Ceph nodes, listening on a specific port and configured with a 
default Ceph ``realm'' and ``zone'' for object storage). To integrate this with 
Kubernetes, we created a Kubernetes Service and Ingress that points to the RGW:  

\begin{itemize}
	\item The \textbf{Service} of type \texttt{ClusterIP} has a backing Endpoints 
	object that contains the static IP of the RGW (e.g., 192.168.132.90, the 
	address of the VM running RGW) and the port (7480 for plain HTTP or 8080 with 
	SSL termination). Essentially, this registers an external service inside 
	Kubernetes.  
	\item The \textbf{Ingress} for RGW uses the host \texttt{rgw.k3s.virtualorfeo.it} 
	(or alternatively \texttt{s3.k3s.virtualorfeo.it} as configured in 
	VirtualOrfeo’s Nginx proxy). We added annotations to allow large uploads 
	(disabling body size limits and request buffering, since experimental data can 
	be quite large). It routes all paths under that host to the RGW service. The 
	ingress is secured with TLS using a certificate issued by the FreeIPA CA.  
	\item This setup means that any application inside the cluster can access the 
	RGW via the internal service DNS 
	(\texttt{http://ceph-rgw.lame-fair.svc.cluster.local:7480}) or via the 
	external URL \texttt{https://rgw.k3s.virtualorfeo.it}. Our Django application 
	uses the external URL because it expects an S3 endpoint with valid TLS.  
\end{itemize}

\medskip

As an alternative to Ceph RGW, VirtualOrfeo also provides the option of deploying 
\textbf{MinIO}, a standalone open-source object storage server that is 
S3-compatible. MinIO is easier to set up for simple tests (it can run as a 
single container). Indeed, our overlay repository contains manifests for a MinIO 
deployment in the cluster.  

The Django application does not need to care whether the underlying storage is 
Ceph or MinIO – it only sees the S3 API. This abstraction is powerful: it means 
we can develop against MinIO locally, then switch to Ceph RGW in integration 
tests, and eventually use the same code with an Amazon S3 bucket in production.

\medskip

Within the object store, our application creates and uses buckets to organize 
data. The data model we follow (as described in Chapter~3) involves Projects, 
Proposals, Samples, and Experiments. We reflect this hierarchy in the storage 
layout:  

\begin{itemize}
	\item Data is partitioned by project: each top-level project 
	(e.g., \texttt{NFFA\_DI}, \texttt{RIANA}) has a dedicated bucket.  
	\item Within each project’s bucket, object keys (paths) encode the proposal, 
	sample, and experiment.  
	\item For example, a raw TIFF image for Experiment~\#123 of Sample~\#5 in 
	Proposal~\texttt{P001} under project \texttt{NFFA\_DI} would be stored as:  
	\begin{verbatim}
		NFFA_DI/PROPOSAL-P001/SAMPLE-5/EXP-123/raw/imagename.tiff
	\end{verbatim}
\end{itemize}

In this scheme, the bucket might be named \texttt{lamefair-data} or 
\texttt{nffa-di-data}, and the directories (prefixes) group the data. The 
application’s code (in the \texttt{pathing.py} and \texttt{s3.py} utilities) 
handles constructing these S3 keys so users do not have to worry about the exact 
paths.

\medskip

After an upload, the system generates a NeXus file (an HDF5-based structured data 
file) containing metadata and possibly processed data. We store these NeXus files 
alongside the raw data or in a parallel location. Common strategies include:  

\begin{itemize}
	\item storing NeXus files in a \texttt{processed/} subdirectory of the same 
	experiment path,  
	\item placing them in a separate bucket for easier access by portals, or  
	\item using a ``mirroring'' strategy, where raw data remains in one bucket 
	while NeXus files are copied to another.  
\end{itemize}

In VirtualOrfeo, we adopted a simple approach: NeXus files are written to the 
same bucket with a distinct prefix (e.g., \texttt{nx/experiment\_123.nxs}). This 
way, the data portal or UI can easily locate them.

\medskip

Using an open, self-describing format like NeXus for processed data aligns with 
best practices in scientific data management, ensuring results are sharable and 
standardized rather than proprietary \parencite{Koennecke2015NeXusFormat, 
	Korir2024TenRecs}. NeXus (built on HDF5) can encapsulate not only the image data 
but also relevant metadata (instrument settings, timestamps, sample identifiers, 
etc.) in one file \parencite{Koennecke2015NeXusFormat}. This ensures that years 
later, users still have a usable record of the experiment.  

This focus on interoperability and reuse reflects the FAIR principles (Findable, 
Accessible, Interoperable, Reusable). By having VirtualOrfeo’s storage mimic the 
final layout, we verified that our application correctly places files and that 
NeXus generation produces files compatible with external tools.

\medskip

To facilitate testing, we also set up some dummy buckets and data. VirtualOrfeo’s 
Ceph RGW can be configured with an initial bucket structure or seeded with 
example data (via Ansible scripts). This was useful for integration tests, such 
as verifying that:  

\begin{itemize}
	\item listing objects shows the expected keys,  
	\item the app detects when a bucket is empty versus containing experiments.  
\end{itemize}

\medskip

In conclusion, VirtualOrfeo’s storage integration provided a realistic 
S3-compatible endpoint. Our application could create buckets, put objects 
(uploads), get objects (downloads), and list contents just as it would on a real 
object store. Using Ceph RGW meant performance and concurrency were closer to a 
production scenario than with a simple local filesystem. It also allowed us to 
test failure modes (e.g., wrong credentials, unreachable store).  

All of this contributes to confidence that the application will handle data 
storage reliably when deployed to the HPC environment or any other S3-compatible 
cloud storage.

\medskip
\noindent With authentication and storage available, the environment is ready 
to host real applications. 
The following subsection describes how a representative Django-based web service 
is packaged and deployed on the VirtualOrfeo cluster.

%%%%%%%%%%%%%%%%%%%%%%%%%%%%%%%%%%%%%%%%%%%%%%%%%%%%%%%%%%%%%%%%%%%%%%%%%%%%%%%%%%%%%

\section{Application Packaging and Deployment on K3s}

To demonstrate how a web application can be deployed in the VirtualOrfeo cluster, 
we use a representative \textbf{Django} application packaged and deployed with 
\textbf{Helm}. The application is built into a container image and distributed 
as a Helm \textbf{chart}, which bundles all Kubernetes manifest templates 
(Deployments, Services, ConfigMaps, Secrets, etc.) along with configuration 
values. This approach ensures that deployments are versioned, reproducible, and 
portable across any Kubernetes cluster.

\medskip

In our repository, the Helm chart is maintained under \texttt{charts/lame-fair/}. 
It defines the following Kubernetes resources, which are typical for deploying a 
Django-based service:  

\begin{itemize}
	\item \textbf{Deployment of the web service:}  
	Creates pods that each run two containers:  
	\begin{itemize}
		\item a container running Django (served by Gunicorn),  
		\item an NGINX sidecar acting as a reverse proxy and serving static files.  
	\end{itemize}  
	This separation allows NGINX to handle static assets efficiently while 
	Gunicorn processes dynamic requests. NGINX also ensures the application is 
	exposed only on standard ports (80/443 through ingress).  
	
	\item \textbf{Deployment of a worker:}  
	Many Django deployments rely on a background task queue (e.g.\ Redis Queue, 
	Celery). To reflect this pattern, the chart defines a second Deployment 
	running a worker container that processes queued jobs, keeping the main web 
	service responsive.  
	
	\item \textbf{Redis instance:}  
	A Redis server (via Bitnami’s sub-chart) is deployed as the queue backend for 
	asynchronous tasks. It runs as a single pod with a PersistentVolume for 
	durability.  
	
	\item \textbf{PostgreSQL database:}  
	A PostgreSQL instance (via Bitnami’s chart) provides the relational database 
	backend. The chart provisions credentials in a Kubernetes Secret and 
	PersistentVolumes for storage. While single-node here, in production one would 
	typically rely on a managed or replicated database.  
	
	\item \textbf{Jobs and hooks:}  
	A Kubernetes Job runs Django’s database migrations on install or upgrade, 
	ensuring the schema is synchronized with the models. This Job is triggered as 
	a Helm hook.  
	
	\item \textbf{Ingress and Service:}  
	An Ingress exposes the service at 
	\texttt{lame-fair.k3s.virtualorfeo.it}, with traffic routed by the cluster’s 
	NGINX ingress controller. Certificates are provisioned by cert-manager via the 
	FreeIPA CA.  
	
	\item \textbf{Configurations:}  
	Settings are supplied through a ConfigMap (for non-sensitive values) and a 
	Secret (for sensitive values such as the Django \texttt{SECRET\_KEY}, OIDC 
	client secrets, and database password).  
\end{itemize}

\medskip

The container image for the Django application is built from a \texttt{Dockerfile} 
and published to a container registry. It includes all dependencies 
(\texttt{requirements.txt}), the application code, and pre-collected static 
files. This image, combined with the Helm chart, provides a portable and 
reproducible deployment unit.

\medskip
\noindent Application packaging, however, is only part of the deployment story. 
Equally important is the secure and flexible management of configuration values 
and secrets, which is handled through Kubernetes-native mechanisms.

%%%%%%%%%%%%%%%%%%%%%%%%%%%%%%%%%%%%%%%%%%%%%%%%%%%%%%%%%%%%%%%%%%%%%%%%%%%%%%%%%%%%%%%%%%%

\section{Secrets and Configuration Management}

Managing configuration and secrets is a critical part of deploying the application. In the VirtualOrfeo environment, the application requires a mix of configuration values: some are non-sensitive (e.g.\ URLs, feature flags), while others are highly sensitive (e.g.\ passwords, private keys, API secrets). Kubernetes provides native resources to handle both types: \textbf{ConfigMaps} for plain configuration and \textbf{Secrets} for sensitive values. Our deployment makes extensive use of both, ensuring that credentials remain secure and configuration stays flexible.

\medskip

At deployment time, we supply several categories of information to the application:

\begin{itemize}
	\item \textbf{Django settings:}  
	Externalized into ConfigMaps rather than baked into the container image.  
	Examples include:  
	\begin{itemize}
		\item whether the app runs in debug mode,  
		\item the allowed hostnames,  
		\item the OIDC callback URL and issuer URL (e.g.\ \texttt{AUTHENTIK\_SERVER\_URL=https://auth.k3s.virtualorfeo.it}).  
	\end{itemize}  
	These are set as environment variables (e.g.\ \texttt{DJANGO\_SETTINGS\_MODULE}, \texttt{DJANGO\_DEBUG}) via the Helm chart’s \texttt{configmap-env.yaml}.
	
	\item \textbf{Secrets for OIDC and database:}  
	\begin{itemize}
		\item OIDC Client ID and Client Secret for the application, generated by Authentik when the app was registered (see Section~\ref{sec:virtualorfeo-iam}).  
		\item PostgreSQL credentials (username, password, host, database name), provisioned by the Helm sub-chart that deploys PostgreSQL. The app reads these values from a Kubernetes Secret (e.g.\ \texttt{pg-postgresql}).  
	\end{itemize}  
	
	\item \textbf{S3 storage credentials and endpoints:}  
	\begin{itemize}
		\item Dedicated S3 user in Ceph RGW (e.g.\ ``django-app-user'') with access and secret keys.  
		\item Keys stored in a Kubernetes Secret (\texttt{S3\_ACCESS\_KEY}, \texttt{S3\_SECRET\_KEY}).  
		\item Endpoint URL (e.g.\ \texttt{https://rgw.k3s.virtualorfeo.it}), treated as non-sensitive and stored in a ConfigMap.  
	\end{itemize}  
	Within Django, these values configure the \texttt{boto3} client.  
	
	\item \textbf{Certificates and CAs:}  
	\begin{itemize}
		\item FreeIPA acts as the internal Certificate Authority.  
		\item To avoid TLS errors when connecting to Authentik or RGW, the FreeIPA CA certificate (PEM) is mounted into the Django container.  
		\item The app is configured (via \texttt{AWS\_CA\_BUNDLE} or Python SSL settings) to trust this certificate.  
		\item The CA certificate itself is stored in a Kubernetes Secret (e.g.\ \texttt{ipa-root-ca}) and mounted at runtime.  
	\end{itemize}  
	
	\item \textbf{Application secrets:}  
	The Django \texttt{SECRET\_KEY}, used for cryptographic signing (sessions, CSRF tokens, etc.), is generated uniquely for each deployment and stored as a Kubernetes Secret.  
\end{itemize}

\medskip

The main reason for using ConfigMaps and Secrets is to separate configuration from code,\footnote{The “Twelve-Factor App” methodology defines twelve best practices for building scalable web applications. Factor III (“Config”) states that configuration such as database credentials or API keys should not be hardcoded in the source but instead supplied at runtime via environment variables or equivalent mechanisms \parencite{Wiggins2011TwelveFactor}.} following best practices for cloud-native deployment.

This avoids hardcoding environment-specific values into the image or repository and aligns with the 12-factor app principle. It also allows credentials to be rotated without changing the application deployment itself. For example, if an S3 key needs replacement, updating the Secret and restarting the pods is sufficient—no rebuild is required.

\medskip

A further consideration is the use of \textbf{scoped credentials}. The S3 access keys created for the application can be restricted to particular permissions:  

\begin{itemize}
	\item In production, one might allow access only to specific buckets or restrict destructive operations (e.g.\ allow \texttt{PUT} but not \texttt{DELETE}).  
	\item In our VirtualOrfeo setup, we allowed full access to a defined namespace of buckets, but finer scoping is possible if needed.  
\end{itemize}

Similarly, OIDC tokens issued by Authentik can be scoped. By default, our app requests only minimal scopes (\texttt{openid}, \texttt{email}, \texttt{profile}) to retrieve basic identity information. Additional scopes (e.g.\ group membership) could be added, but limiting scopes to what is necessary is a good security practice.

\medskip

This principle of scoped access also applies at the application level:  

\begin{itemize}
	\item A frontend designed only for browsing data could use read-only S3 keys or tokens.  
	\item Our current application handles both uploads and browsing, but a future read-only data portal could be configured with narrower permissions.  
\end{itemize}

\medskip

In summary, the deployment relies on Kubernetes-native mechanisms for securely injecting configuration and secrets. By distinguishing clearly between sensitive and non-sensitive values, we lower the risk of leakage and make configuration easier to manage. Helm charts were written to ensure that secret values are never displayed in logs or upgrade diffs, following common cloud-native security practices.
