
%%%%%%%%%%%%%%%%%%%%%%%%%%%%%%%%%%%%%%%%%%%%%%%%%%%%
%
% Adapted to English and filled with thesis content
% for the University of Trieste theme by Isac Pasianotto.
%
%%%%%%%%%%%%%%%%%%%%%%%%%%%%%%%%%%%%%%%%%%%%%%%%%%%%

\documentclass{beamer}

% Load all required packages for the theme
\usepackage{otherResources/presentazione_allPackages}

% Document metadata
\hypersetup{
    pdfauthor={Nicola Perin},
    pdftitle={Designing and deploying a FAIR-by-design data pipeline and platform for electron microscopy laboratories},
    pdfsubject={Research thesis in: Data Management}
}

% Title block
\title[FAIR-by-design EM pipeline]{Designing and deploying a FAIR-by-design data pipeline and platform for electron microscopy laboratories}
\subtitle{Research thesis in: Data Management}
\institute{University of Trieste}
\author[Nicola Perin]{Nicola Perin}
\year=2025
\month=09
\day=19

% Advisors & labels (translated to English)
\def\relatore{Dott. Federica Bazzocchi}
\def\relatoreLabel{Supervisor}
% \def\correlatore{<Co-supervisor name>}
% \def\correlatoreLabel{Co-supervisor}
\def\candidatoLabel{Candidate}

% Logos (use your local paths if different)
\def\LogoUniversita{otherResources/Units.Logo3Righe.png}
\def\LogoDipartimento{otherResources/DSSC_logo.png}
\def\LogoFiligrana{otherResources/background-blu.png}

% Apply theme customizations
\input{otherResources/presentazione_myThemeSetting.tex}

% ---- Slides ----
\begin{document}
	
	% =====================
	% Title & Agenda
	% =====================
	
	% Title frame (special dissertation style from the theme)
	\begin{frame}
		\setTitlestyleDissertation
		\maketitle
	\end{frame}
	
	% Agenda
	\section*{Agenda}
	\begin{frame}
		\frametitle{Agenda}
		\tableofcontents
	\end{frame}
	
	% =====================
	% Context & Goals
	% =====================
	\section{Context \& Goals}
	
	\begin{frame}
		\frametitle{Motivation \& Problem}
		\begin{itemize}
			\item Electron microscopy (EM) labs produce very large and mixed datasets (images, diffraction patterns, spectra).
			\item Proprietary file formats and little metadata make data hard to share and reuse.
			\item Informal methods (like file names or personal notes) do not work well in collaborations.
		\end{itemize}
		\vspace{1em}
		\textbf{Challenge:} How can we make this data easier to share, reuse, and preserve?
	\end{frame}
	
	\begin{frame}
		\frametitle{FAIR Principles}
		\centering
		\includegraphics[width=0.7\textwidth]{otherResources/FAIR_data_principles.png}
		
		\vspace{1em}
		\small
		Answer: apply the FAIR principles — make microscopy data  
		\textbf{Findable, Accessible, Interoperable, Reusable}.  
		Emphasis on metadata, standardized formats, and machine-actionable records.
	\end{frame}

	\begin{frame}
		\frametitle{Standards \& Formats}
		\begin{columns}
			\column{0.55\textwidth}
			\includegraphics[width=\textwidth]{img/diagrams/nexus_nxem.png}
			\column{0.45\textwidth}
			\begin{itemize}
				\item \textbf{HDF5}: hierarchical container for large arrays + attributes.
				\item \textbf{NeXus}: scientific conventions (\texttt{NXinstrument}, \texttt{NXsample}).
				\item \textbf{NXem}: EM definition (images, diffraction, EDS/EELS, 4D-STEM).
			\end{itemize}
		\end{columns}
	\end{frame}
	
	\begin{frame}
		\frametitle{Institutional Context}
		\begin{columns}
			\column{0.5\textwidth}
			\includegraphics[width=\textwidth]{img/diagrams/area_science_park.png}
			\column{0.5\textwidth}
			Area Science Park → RIT institute with labs:  
			\begin{itemize}
				\item LADE (data engineering)  
				\item LAGE (genomics)  
				\item LAME (electron microscopy)  
			\end{itemize}
			\medskip
			ORFEO data center provides HPC, Ceph storage, and identity services.  
			\medskip
			This project targets LAME workflows, but is scalable across NFFA–DI.
		\end{columns}
	\end{frame}

\end{document}