
%%%%%%%%%%%%%%%%%%%%%%%%%%%%%%%%%%%%%%%%%%%%%%%%%%%%
%
% Adapted to English and filled with thesis content
% for the University of Trieste theme by Isac Pasianotto.
%
%%%%%%%%%%%%%%%%%%%%%%%%%%%%%%%%%%%%%%%%%%%%%%%%%%%%

\documentclass{beamer}

% Load all required packages for the theme
\usepackage{otherResources/presentazione_allPackages}

% Document metadata
\hypersetup{
    pdfauthor={Nicola Perin},
    pdftitle={Designing and deploying a FAIR-by-design data pipeline and platform for electron microscopy laboratories},
    pdfsubject={Research thesis in: Data Management}
}

% Title block
\title[FAIR-by-design EM pipeline]{Designing and deploying a FAIR-by-design data pipeline and platform for electron microscopy laboratories}
\subtitle{Research thesis in: Data Management}
\institute{University of Trieste}
\author[Nicola Perin]{Nicola Perin}
\year=2025
\month=09
\day=19

% Advisors & labels (translated to English)
\def\relatore{Dott. Federica Bazzocchi}
\def\relatoreLabel{Supervisor}
% \def\correlatore{<Co-supervisor name>}
% \def\correlatoreLabel{Co-supervisor}
\def\candidatoLabel{Candidate}

% Logos (use your local paths if different)
\def\LogoUniversita{otherResources/Units.Logo3Righe.png}
\def\LogoDipartimento{otherResources/DSSC_logo.png}
\def\LogoFiligrana{otherResources/background-blu.png}

% Apply theme customizations
%%%%%%%%%%%%%%%%%%%%%%%%%%%%%%%%%%%%%%%%%%%%%%%%%%%%%%%%%%%
%
% Copyright 2022 by Isac Pasianotto
%
% This file may be distributed and/or modified
%
% 1. under the LaTeX Project Public License and/or
% 2. under the GNU Public License.
%
%%%%%%%%%%%%%%%%%%%%%%%%%%%%%%%%%%%%%%%%%%%%%%%%%%%%%%%%%%%

%% 	Variabili di tipo "color": 

\definecolor{bluUnits100}{rgb}{0.16,0.22,0.36} 
\definecolor{bluUnits80}{rgb}{0.22,0.3,0.51}
\definecolor{bluUnits70}{rgb}{0.25,0.35,0.58}
\definecolor{bluUnits50}{rgb}{0.41,0.51,0.74}
\definecolor{bluUnits40}{rgb}{0.56,0.63,0.81}
\definecolor{bluUnits25}{rgb}{0.78,0.82,0.9}
\definecolor{bluUnits10}{rgb}{0.93,0.94,0.96}
\definecolor{grey}{rgb}{0.3686, 0.5255, 0.6235} 

%%	Palette di colori 

\setbeamercolor{palette primary}{bg=bluUnits100,fg=white}
\setbeamercolor{palette secondary}{bg=bluUnits80,fg=bluUnits25}
\setbeamercolor{palette tertiary}{bg=bluUnits50,fg=bluUnits100}
\setbeamercolor{palette quaternary}{bg=bluUnits40,fg=white}
\setbeamercolor{palette light primary}{bg=bluUnits25,fg=bluUnits100}
\setbeamercolor{palette titleframe}{bg=bluUnits10, fg=bluUnits80}


		%%%%%%%%%%%%%%%%%%%%%%%%%%%%%%%%%%
		%% Impostazioni generali slide  %%
		%%%%%%%%%%%%%%%%%%%%%%%%%%%%%%%%%%

%%	Setta l'immagine da mettere come sfondo, riducendone l'opacità
\usebackgroundtemplate{\tikz\node[opacity=0.1]{\includegraphics[height=\textheight]{\LogoFiligrana}};}

%%	Elenchi puntati, numerati, etc.

\setbeamercolor{structure}{fg=bluUnits80}
\setbeamertemplate{enumerate item}[circle]
\setbeamertemplate{itemize subitem}[ball]
% Valutare a secoda del contesto se sostituire con 
% \setbeamertemplate{items}[circle]
\setbeamercolor{alerted text}{fg=bluUnits50}

%% 	Colore delle scritte nella presentazione

\setbeamercolor{normal text}{fg=bluUnits100,bg=white}

%% 	Settaggio della linea in alto (headline)

\setbeamertemplate{headline}{
	\vskip1pt
	\leavevmode	
	\hbox{
		\begin{beamercolorbox}[wd=.99\paperwidth,ht=2.5ex,dp=1.125ex]{palette light primary}
			\insertsectionnavigationhorizontal{\paperwidth}{}{\hskip0pt plus1filll}
		\end{beamercolorbox}
	}
}

%%	Settaggio riga in basso (footline) 
 
\setbeamertemplate{footline}{
       \leavevmode
	   \hbox{
           \begin{beamercolorbox}[wd=.2\textwidth,ht=2.6ex,dp=1ex,center]{palette tertiary}
		    \usebeamerfont{author in head/foot}\insertshortauthor
    	\end{beamercolorbox}
	
    	\begin{beamercolorbox}[wd=.27\textwidth,ht=2.6ex,dp=1ex,center]{palette quaternary}
	   		\usebeamerfont{institute in head/foot}\insertshortinstitute
	   	\end{beamercolorbox}
	
    	\begin{beamercolorbox}[wd=.40\textwidth,ht=2.6ex,dp=1ex,center]{palette primary}
    		\usebeamerfont{title in head/foot}\insertshorttitle
    	\end{beamercolorbox}

    	\begin{beamercolorbox}[wd=.1\textwidth,ht=2.6ex,dp=1ex,center]{palette light primary}
	   		\insertframenumber{}/\inserttotalframenumber
	    \end{beamercolorbox}
    }
    \vskip2pt

}

%%	Settaggio tittoli delle slide  

\setbeamertemplate{frametitle}{
	\begin{beamercolorbox}[wd=\paperwidth,ht=2.75ex,dp=1ex,left]{palette titleframe}
		\qquad \textbf{\insertframetitle}
	\end{beamercolorbox}
}


		%%%%%%%%%%%%%%%%%%%%%%%%%%%%%%%
		%% Impostazioni Prima Slide  %%
		%%%%%%%%%%%%%%%%%%%%%%%%%%%%%%%
	
	
	
	
		
\def\setTitlestyleDissertation{
	
	\defbeamertemplate*{title page}{customized}[1][]{
		
		%  Commentare il seguente ambiente {center} e decommentare {flushright} quello successivo in caso
		%	si voglia usare solo il logo dell'UNI
		
		\begin{center}
			\begin{multicols}{2}
				\includegraphics[width=0.45\textwidth]{\LogoDipartimento}
			\columnbreak
				\includegraphics[width=0.45\textwidth]{\LogoUniversita}		
			\end{multicols}
		\end{center}
	
		%	\begin{flushright}
		%		\includegraphics[width=0.45\textwidth]{\LogoUniversita}	
		%	\end{flushright}
	
		\smallskip
		
		\begin{center}		
			\usebeamerfont{title}\textbf{\inserttitle}\par
			\usebeamerfont{subtitle}\usebeamercolor[fg]{subtitle}\insertsubtitle\par
			\medskip		
			
			%% Il seguente layout dentro l'ambiente multicols serve per le tesi.
			
			\begin{multicols}{2}
				\begin{tabular}{c}
					\usebeamerfont{normal text}{\relatoreLabel} \\
					\usebeamerfont{author}{\relatore}
					
						% Decommentare in caso siano presenti dei correlatori
					%\\
					%\usebeamerfont{normal text}{\correlatoreLabel} \\
					%\usebeamerfont{author}{\correlatore}
						
				\end{tabular}					
				\columnbreak
				\begin{tabular}{c}
					\candidatoLabel \\
					\usebeamerfont{author}{\insertauthor}
				\end{tabular}
			\end{multicols}
		
			\par
			
			\bigskip  	% --> nel caso di relatore e basta
			%\smallskip 	% --> nel caso di relatore + correlatore
			
			\insertinstitute\par
			
			\bigskip	% --> nel caso di relatore e basta
			%\smallskip	% --> nel caso di relatore + correlatore
			
			\usebeamerfont{date}\insertdate\par
			
			\bigskip	% --> nel caso di relatore e basta
			%\smallskip	% --> nel caso di relatore + correlatore
		\end{center}
	}
}


% ---- Slides ----
\begin{document}

% Title frame (special dissertation style from the theme)
\begin{frame}
\setTitlestyleDissertation
\maketitle
\end{frame}

% Table of contents
\section*{Agenda}
\begin{frame}
    \frametitle{Agenda}
    \tableofcontents
\end{frame}

% =====================
% Context & Goals
% =====================
\section{Context \& Goals}

\begin{frame}
    \frametitle{Motivation \& Problem}
    \begin{itemize}
        \item Electron microscopy (EM) labs generate very large, heterogeneous datasets (images, diffraction, spectra).
        \item Vendor-specific formats plus sparse metadata hinder interoperability \& reuse.
        \item Informal practices (file names, local notes) do not scale across collaborations.
    \end{itemize}
\end{frame}

\begin{frame}
    \frametitle{FAIR Principles (Goal)}
    \begin{itemize}
        \item Make data \textbf{Findable, Accessible, Interoperable, Reusable}.
        \item Emphasize rich metadata, standardized formats, and machine actionability.
        \item Align with funder \& journal expectations; enable open science.
    \end{itemize}
\end{frame}

\begin{frame}
    \frametitle{Standards \& Formats}
    \begin{itemize}
        \item \textbf{HDF5}: performant hierarchical container for large arrays and attributes.
        \item \textbf{NeXus}: community conventions (e.g., \texttt{NXinstrument}, \texttt{NXsample}).
        \item \textbf{NXem}: EM application definition (images, diffraction, EDS/EELS, 4D-STEM) + metadata.
    \end{itemize}
\end{frame}

\begin{frame}
    \frametitle{Institutional Context}
    \begin{itemize}
        \item Area Science Park → RIT institute with labs: LADE (data engineering), LAGE (genomics), LAME (electron microscopy).
        \item ORFEO data center provides HPC, Ceph storage, and identity services.
        \item Project targets LAME workflows; scalable across NFFA–DI.
    \end{itemize}
\end{frame}

% =====================
% Infrastructure & Platform
% =====================
\section{Infrastructure \& Platform}

\begin{frame}
	\frametitle{Infrastructure: ORFEO}
	\begin{itemize}
		\item \textbf{Ceph} distributed storage with tiers for speed/capacity; replication and erasure coding.
		\item \textbf{RADOS Gateway (RGW)}: S3-compatible interface for object storage.
		\item \textbf{Identity and SSO}: FreeIPA (directory \& CA) + Authentik (OIDC provider).
		\item Integrated into \textbf{ORFEO HPC cluster}: compute + storage + identity under one umbrella.
	\end{itemize}
\end{frame}

\begin{frame}
	\frametitle{Application Platform}
	\begin{itemize}
		\item Web app streamlines \textbf{upload + annotation} in one flow.
		\item Built on \textbf{Django (MVT)} + PostgreSQL; REST API for scripted ingest.
		\item Containerized services deployed with \textbf{Helm charts}.
		\item Separation of interactive deposit (at the lab) from heavy processing (at data center).
	\end{itemize}
\end{frame}

\begin{frame}
	\frametitle{Deployment \& Validation}
	\begin{itemize}
		\item \textbf{VirtualOrfeo}: digital twin of ORFEO HPC, safe for experimentation.
		\item Built from VMs (KVM/QEMU) managed via Vagrant + Ansible.
		\item Replicates: directory services, Kubernetes, Ceph cluster.
		\item Enables fast iteration without impacting production.
	\end{itemize}
\end{frame}

\begin{frame}
	\frametitle{K3s Cluster Topology (VirtualOrfeo)}
	\begin{itemize}
		\item Lightweight \textbf{K3s} Kubernetes with one control-plane/worker VM.
		\item Namespaces: \texttt{authentik}, \texttt{lame-fair}, monitoring.
		\item Ingress via \textbf{NGINX} + load balancing with \textbf{MetalLB}.
		\item Certificates issued by FreeIPA CA, automated by \textbf{cert-manager}.
	\end{itemize}
\end{frame}

\begin{frame}
	\frametitle{Identity \& Access Management}
	\begin{itemize}
		\item FreeIPA provides directory, groups, CA.
		\item Authentik acts as OIDC provider integrated with FreeIPA.
		\item Django app registered as OIDC client (\texttt{lame-fair}).
		\item Secure login: Authentik manages tokens; app never handles raw passwords.
	\end{itemize}
\end{frame}

\begin{frame}
	\frametitle{Storage Integration in VirtualOrfeo}
	\begin{itemize}
		\item \textbf{Ceph RGW}: S3-compatible object store for raw + processed data.
		\item Optional MinIO for lightweight tests.
		\item Bucket layout mirrors Project $\rightarrow$ Proposal $\rightarrow$ Sample $\rightarrow$ Experiment.
		\item NeXus files generated alongside raw TIFFs for FAIR compliance.
	\end{itemize}
\end{frame}

\begin{frame}
	\frametitle{Application Deployment on K3s}
	\begin{itemize}
		\item Packaged as a \textbf{Helm chart} (pods, services, secrets, ingress).
		\item Gunicorn + NGINX sidecar for Django web service.
		\item Background worker with Redis queue.
		\item PostgreSQL database + migrations as Helm hooks.
	\end{itemize}
\end{frame}


% =====================
% Design & Implementation
% =====================
\section{Design \& Implementation}

\begin{frame}
    \frametitle{Domain Model \& Data Flow}
    \begin{itemize}
        \item Project → Proposal → Sample → Experiment → Measurement hierarchy.
        \item Metadata captured early; mapped into NeXus/NXem.
        \item Curated outputs stored in centralized data lake (S3/Ceph).
    \end{itemize}
\end{frame}

\begin{frame}
    \frametitle{Metadata \& NeXus Construction}
    \begin{itemize}
        \item Automated mapping of instrument settings (beam energy, detectors, stage coordinates).
        \item Human-readable README + machine-readable NeXus containers.
        \item Libraries (e.g., \texttt{pynxtools-em}) lower the adoption barrier.
    \end{itemize}
\end{frame}

\begin{frame}
    \frametitle{Storage Gateway \& Background Tasks}
    \begin{itemize}
        \item Upload pipeline writes to object storage; lifecycle managed server-side.
        \item Background workers handle conversion, validation, indexing.
        \item Auditability \& provenance preserved end-to-end.
    \end{itemize}
\end{frame}

\begin{frame}
    \frametitle{Security, Performance, Scalability}
    \begin{itemize}
        \item OIDC tokens with group claims → role-based access in-app.
        \item Ceph scales horizontally; PostgreSQL tuned for concurrent users.
        \item Asynchronous jobs decouple ingestion from heavy compute.
    \end{itemize}
\end{frame}

% =====================
% Results & Conclusions
% =====================
\section{Results \& Conclusions}

\begin{frame}
    \frametitle{Contributions}
    \begin{itemize}
        \item FAIR-by-design workflow from acquisition to curated NeXus/NXem.
        \item Unified upload + annotation web app; API surface for automation.
        \item Deployment blueprint validated in VirtualOrfeo; ready for ORFEO/NFFA–DI.
    \end{itemize}
\end{frame}

\begin{frame}
    \frametitle{Conclusions \& Next Steps}
    \begin{itemize}
        \item Reproducible, interoperable EM data pipeline proved feasible.
        \item Scale to additional instruments and labs; expand validators \& viewers.
        \item Prepare for open access portals and cross-lab discovery.
    \end{itemize}
\end{frame}

\end{document}
