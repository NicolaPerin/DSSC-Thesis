\chapter*{Conclusions}\label{chap:conclusions}
% Do not edit
\addcontentsline{toc}{chapter}{Conclusion}
\markboth{Conclusion}{Conclusion}

This thesis investigated how the FAIR data principles can be applied to the 
daily work of an electron microscopy laboratory. The problem addressed was the 
increasing amount and complexity of data produced by modern instruments, and the 
difficulty of storing, organizing, and reusing these data in a systematic way. 
The goal was to design and implement a data pipeline and platform that make 
microscopy datasets easier to manage and more useful in the long term.

The work resulted in a solid and FAIR-by-design pipeline that starts from the moment of 
data acquisition. Raw files are placed in an object store, enriched with 
metadata, and where possible converted into standardized NeXus/NXem files. The 
system records checksums and README files to keep track of provenance, and 
separates raw from processed data using dedicated buckets. All of this is 
integrated into a web application that allows researchers to create projects, 
proposals, samples, and experiments, and to upload and browse their data. 
Background workers handle the heavier tasks such as checksum calculation and 
file conversion, so that the web interface remains responsive.  

To ensure that the platform could be deployed safely in production, all components were first developed and validated in the \emph{VirtualOrfeo} environment, a digital twin of the ORFEO cluster. This made it possible to test authentication, storage, and deployment choices without risk, before moving to the real infrastructure. The layered software design — with clear separation between the domain model, storage gateway, background tasks, and user interfaces — further contributes to robustness and maintainability. Overall, the work demonstrates that a FAIR-by-design pipeline for electron microscopy can be deployed on ORFEO, the high-performance computing and data center operated by Area Science Park in Trieste, providing a reproducible and future-proof solution for managing scientific data.

There are several areas for future work. The NeXus writers can be extended to 
validate files more strictly, convert units into standard forms, and support 
multi-frame datasets. More instrument mappings should be added so that the 
system can be used with a wider variety of microscopes. Connections to external 
research infrastructures, including persistent identifiers and metadata 
standards, would improve the visibility and reuse of the data. Finally, scaling 
up the system in production will require monitoring how it behaves under higher 
data rates and adjusting the storage and queueing backends as needed.

Although this work focused on the electron microscopy workflows of LAME, its 
design is not specific to a single technique. The same architecture — combining 
structured uploads, object storage, NeXus-based metadata handling, and 
background processing — can be adapted to other scientific domains such as 
spectroscopy, tomography, or genomics, and reused across laboratories both 
inside and outside Area Science Park. By keeping the components modular and 
standards-based, the pipeline offers a general blueprint for FAIR data 
management beyond microscopy.

The software stack developed in this thesis is openly released on 
GitHub\footnote{\url{https://github.com/NicolaPerin/fair-em-pipeline}} under the permissive 
\emph{MIT License}. This allows other researchers to freely adopt, extend, and 
integrate the platform into their own workflows, ensuring that the results of 
this thesis contribute not only to local needs but also to the broader goals of 
open science and reproducible research. In this sense, the project applies the 
FAIR principles not only to scientific data but also to the tools that manage 
them.

In summary, this thesis shows that FAIR data management for electron microscopy 
is possible with a pipeline and platform that combine standardized formats, 
automated processing, and integration with laboratory workflows. The result is a 
practical approach that supports both local research needs and the broader 
objectives of scientific infrastructures, positioning ORFEO and Area Science Park 
at the forefront of open and reproducible science in Europe.

