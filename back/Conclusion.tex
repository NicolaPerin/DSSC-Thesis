\chapter*{Conclusions}\label{chap:concl}
% Do not edit
\addcontentsline{toc}{chapter}{Conclusion}
\label{chap:concl}
\markboth{Conclusion}{Conclusion}

This thesis investigated how the FAIR data principles can be applied to the 
daily work of an electron microscopy laboratory. The problem addressed was the 
increasing amount and complexity of data produced by modern instruments, and the 
difficulty of storing, organizing, and reusing these data in a systematic way. 
The goal was to design and implement a data pipeline and platform that make 
microscopy datasets easier to manage and more useful in the long term.

The work resulted in a FAIR-by-design pipeline that starts from the moment of 
data acquisition. Raw files are placed in an object store, enriched with 
metadata, and where possible converted into standardized NeXus/NXem files. The 
system records checksums and README files to keep track of provenance, and 
separates raw from processed data using dedicated buckets. All of this is 
integrated into a web application that allows researchers to create projects, 
proposals, samples, and experiments, and to upload and browse their data. 
Background workers handle the heavier tasks such as checksum calculation and 
file conversion, so that the web interface remains responsive.  

To make sure the platform could be deployed safely in production, the work also 
set up and used the VirtualOrfeo environment, which reproduces the main services 
of the ORFEO cluster. This allowed testing of authentication, storage, and 
deployment choices before moving to the real infrastructure. A layered software 
design helped keep the different parts of the system — domain model, storage 
gateway, background tasks, and user interfaces — clear and independent.

There are several areas for future work. The NeXus writers can be extended to 
validate files more strictly, convert units into standard forms, and support 
multi-frame datasets. More instrument mappings should be added so that the 
system can be used with a wider variety of microscopes. Connections to external 
research infrastructures, including persistent identifiers and metadata 
standards, would improve the visibility and reuse of the data. Finally, scaling 
up the system in production will require monitoring how it behaves under higher 
data rates and adjusting the storage and queueing backends as needed.

In summary, the thesis shows that FAIR data management for electron microscopy 
is possible with a pipeline and platform that combine standardized formats, 
automated processing, and integration with laboratory workflows. The result is a 
practical approach that supports both local research needs and the broader goals 
of open and reproducible science.
